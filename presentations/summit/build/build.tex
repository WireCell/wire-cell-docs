% TEXINPUTS=.:$HOME/git/bvtex: latexmk  -pdf <main>.tex
\documentclass[xcolor=dvipsnames]{beamer}

\input{defaults}
\input{beamer/preamble}

\setbeamertemplate{navigation symbols}{}
% \setbeamertemplate{background}[grid][step=1cm]

\usepackage{siunitx}
\usepackage{xmpmulti}
\usepackage[export]{adjustbox}

\usepackage[outline]{contour}
\usepackage{tikz}
\usetikzlibrary{shapes.geometric, arrows}
\usetikzlibrary{positioning}

\definecolor{bvtitlecolor}{rgb}{0.98, 0.92, 0.84}
\definecolor{bvoutline}{rgb}{0.1, 0.1, 0.1}

\renewcommand{\bvtitleauthor}{Brett Viren}
\renewcommand{\bvtit}{Building Wire Cell Software}
\renewcommand{\bvtitle}{\LARGE Building Wire Cell Software\\(prototype and toolkit)}
\renewcommand{\bvevent}{Wire Cell Seminar 2015 Dec 7-9}
\renewcommand{\bvbeamerbackground}{}

% http://tex.stackexchange.com/a/23550
\makeatletter
\providecommand{\beamer@slideinframe}{0}%
\tikzset{highlight/.code={\ifnum#1=\beamer@slideinframe \tikzset{draw=red,text=red}\fi},highlight/.value required}
\makeatother

\setbeamertemplate{headline}{%
\leavevmode%
  \hbox{%
    \begin{beamercolorbox}[wd=\paperwidth,ht=2.5ex,dp=1.125ex]{palette quaternary}%
    \insertsectionnavigationhorizontal{\paperwidth}{}{\hskip0pt plus1filll}
    \end{beamercolorbox}%
  }
}
\begin{document}
\input{beamer/title.tex}
\input{beamer/toc.tex}

\lstset{%
  language=C++,
                basicstyle=\scriptsize\ttfamily,
                keywordstyle=\color{blue}\ttfamily,
                stringstyle=\color{red}\ttfamily,
                commentstyle=\color{gray}\ttfamily,
                morecomment=[l][\color{magenta}]{\#}
}

\section{Prerequisites}

\begin{frame}
  \frametitle{Major Prerequisites}

  Wire Cell is fairly trivial to install but it has a few ``heavy'' prerequisites. 

  \begin{itemize}
  \item C++11 compiler: GCC ($\ge 4.9$), Clang ($>=7$)
  \item BOOST ($\ge 1.55$)
  \item ROOT ($\ge 6.02$)
  \end{itemize}
\end{frame}

\subsection{Automated builds of externals}
\begin{frame}
  \frametitle{Automated Build of Externals}

  Wire Cell Prototype provides
  \href{https://github.com/brettviren/worch}{Worch}-based build
  automation for the externals.  

  \begin{itemize}
  \item Maybe hasn't kept up.
  \item There are so few external packages that it may be more hassle than it's worth.
  \end{itemize}

  For details, see:

  \begin{center}
      \url{https://github.com/BNLIF/wire-cell-externals}
  \end{center}
\end{frame}

\subsection{Guidance for manual builds of externals}

\begin{frame}
  \frametitle{Compiler}
  \begin{itemize}
  \item Wire Cell uses C++11 and maybe some C++14 sneaking in. 
  \item Most decent OSes provide this but your Sci. Linux box may not.
    \begin{itemize}
    \item Debian world has had GCC 4.9.2+ for a while.
    \item Recent Mac has Clang 7.
    \item RHEL 7 has GCC 4.8, native - untested.
    \end{itemize}
  \item It's up to you to provide a working compiler.
    \begin{itemize}
    \item Use system compiler if it's sufficient.
    \item Build GCC from source.
    \end{itemize}
  \end{itemize}
\end{frame}

\begin{frame}[fragile]
  \frametitle{ROOT6 - from source}
  Configure for C++11, FFTW3, Minuit and Python.

\scriptsize
\begin{verbatim}
  wget https://root.cern.ch/download/root_v6.05.02.source.tar.gz
  tar -xvf root_v6.05.02.source.tar.gz
  mkdir root-build
  cd root-build/
  cmake -DCMAKE_INSTALL_PREFIX={install_dir} \
	-Dbuiltin_xrootd=ON -Dcxx11=ON -Dpythia6=OFF -Dminuit2=ON \
	-Dpython=ON \
	-DFFTW_INCLUDE_DIR=/usr/include \
	-DFFTW_LIBRARY=/usr/lib/x86_64-linux-gnu/libfftw3.so \
	../root/
  make
  make install
\end{verbatim}

\small
For this manual installation, activate ROOT environment using the (ROOT) standard:

\begin{verbatim}
$ source /path/to/install/bin/thisroot.sh
\end{verbatim}

\end{frame}

\subsection{Leveraging your OS packaging}

\begin{frame}[fragile]
  \frametitle{Debian and co.}

  Recent Debian (Ubuntu, etc) can provide everything except ROOT6:
\begin{verbatim}
sudo apt-get install git build-essential gcc g++ make cmake \
    python-dev libboost-all-dev \
    libsqlite3-dev tcl8.5-dev libxpm-dev libxft-dev libXext-dev \
    libeigen3-dev
\end{verbatim}
  Maybe others are needed...
\end{frame}

\begin{frame}[fragile]
  \frametitle{Mac OS X + Homebrew}
  On Mac (tested on 10.10.5, ``Yosemite'') you get everything for cheap with \href{http://brew.sh/}{homebrew}!
\begin{verbatim}
brew install python
brew install boost --with-python
brew install homebrew/science/root6
\end{verbatim}

  If you are new to homebrew see \url{http://brew.sh/}.

  The native Mac compiler is now Clang but with the command confusingly still named ``\texttt{g++}''.


\end{frame}

\subsection{Site-specific}

\begin{frame}[fragile]
  \frametitle{Fermilab UPS}
  This worked at one time to provide ROOT and BOOST:
\begin{verbatim}
$ source /grid/fermiapp/products/larsoft/setups
$ setup root v6_04_02 -q e7:prof
$ setup boost v1_57_0 -q e7:prof
\end{verbatim}

\end{frame}

\begin{frame}[fragile]
  \frametitle{BNL RACF}
  A build of ROOT 6 is available on RACF:

\begin{verbatim}
$ source /gpfs01/lbne/users/sw/wc/bin/thisroot.sh
\end{verbatim}

This will also set up to use \texttt{g++} 4.9.2.

\end{frame}

\section{Wire Cell Prototype Software}

\subsection{Repositories}

\begin{frame}
  \frametitle{Prototype Software Repositories}

  All wire cell \textbf{prototype} source is in the ``BNLIF'' GitHub organization:

  \begin{center}
    \url{https://github.com/BNLIF/}
  \end{center}

  Git submodules are used to aggregate each package from this top level one:

  \begin{center}
    \url{https://github.com/BNLIF/wire-cell}
  \end{center}

  By default, git submodules are hooked in assuming SSH access (not HTTPS).

\end{frame}

\begin{frame}[fragile]
  \frametitle{Cloning the source}
  Developers do:
\begin{verbatim}
$ git clone git@github.com:BNLIF/wire-cell.git
$ cd wire-cell/
\end{verbatim}

  Others may do:
\begin{verbatim}
$ git clone https://github.com/BNLIF/wire-cell.git
$ cd wire-cell/
$ ./switch-git-urls
\end{verbatim}

  Everyone does:

\begin{verbatim}
$ git submodule init
$ git submodule update
\end{verbatim}

\end{frame}

\subsection{Building}

\begin{frame}[fragile]
  \frametitle{Building Wire Cell Prototype software}
  
  Wire Cell Prototype software is built with \textbf{waf} using the
  provided \texttt{waf-tools} package:

\begin{verbatim}
$ alias waf=`pwd`/waf-tools/waf
$ waf --prefix=/path/to/install \
        configure build install
\end{verbatim}


\vspace{5mm}
That's it.

\end{frame}

\subsection{Run-time Environment}

\begin{frame}[fragile]
  \frametitle{Run-time Environment}

  Externals must be setup in whatever manner is associated with how they were built.

  For Wire Cell itself one needs to set the usual:

  \begin{itemize}
  \item \verb|PATH|
  \item \verb|LD_LIBRARY_PATH|
  \item \verb|PYTHONPATH|
  \end{itemize}

  to point to appropriate directories under \verb|/path/to/install|.

\end{frame}

\section{Wire Cell Toolkit Software}

\begin{frame}
  \tableofcontents[currentsection,hideothersubsections]
\end{frame}

\subsection{Externals}

\begin{frame}
  \frametitle{Externals for Wire Cell Toolkit}

  The Wire Cell Toolkit requires both fewer and more externals than the prototype:

  \begin{itemize}
  \item C++11 compiler: GCC ($\ge 4.9$), Clang ($>=7$)
  \item BOOST ($\ge 1.55$)
  \end{itemize}

  For parallel functionality:

  \begin{itemize}
  \item Intel TBB ($\ge 4.3$)
  \end{itemize}

  Some tests, I/O and visualization packages require:

  \begin{itemize}
  \item ROOT ($\ge 6.02$)
  \end{itemize}

  Same guidance for providing externals for the WC-Prototype applies to the WC-Toolkit.

\end{frame}


\subsection{Repositories}

\begin{frame}[fragile]
  \frametitle{Wire Cell Toolkit Repositories}

  All wire cell \textbf{toolkit} source is in the ``WireCell'' GitHub organization:

  \begin{center}
    \url{https://github.com/WireCell/}
  \end{center}

  Git submodules are used to aggregate each package from this top level one:

  \begin{center}
    \url{https://github.com/WireCell/wire-cell-build}
  \end{center}

  Same guidance for getting the WC-Prototype source applies to the WC-Toolkit.

\footnotesize
\begin{verbatim}
$ git clone git@github.com:WireCell/wire-cell-build.git
$ cd wire-cell-build/
$ ./switch-git-urls  # <-- if not a developer
$ git submodule init
$ git submodule update
\end{verbatim}

\end{frame}

\subsection{Building}

\begin{frame}[fragile]
  \frametitle{Building Wire Cell Toolkit software}
  
  Wire Cell Toolkit software is built with \textbf{waf} using the
  provided waf-bundled command \textbf{wcb}.

\begin{verbatim}
$ ./wcb --prefix=/path/to/install \
        configure build install
\end{verbatim}

If your dependencies are not found by \texttt{wcb} then:

\begin{verbatim}
$ ./wcb --help
$ ./wcb --boost-libs=... --boost-includes=... \
        --with-root=... configure
$ ./wcb
$ ./wcb install
\end{verbatim}

\end{frame}

\subsection{Run-time Environment}

\begin{frame}[fragile]
  \frametitle{Run-time Environment}

  Externals must be setup in whatever manner is associated with how they were built.

  For Wire Cell itself one needs to set the usual:

  \begin{itemize}
  \item \verb|PATH|
  \item \verb|LD_LIBRARY_PATH|
  \item \verb|PYTHONPATH|
  \end{itemize}

  to point to appropriate directories under \verb|/path/to/install|.

\end{frame}

\end{document}


%%% Local Variables:
%%% mode: latex
%%% TeX-master: t
%%% End:
