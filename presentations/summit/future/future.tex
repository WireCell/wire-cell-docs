% TEXINPUTS=.:$HOME/git/bvtex: latexmk  -pdf <main>.tex
\documentclass[xcolor=dvipsnames]{beamer}

\input{defaults}
\input{beamer/preamble}

\setbeamertemplate{navigation symbols}{}
% \setbeamertemplate{background}[grid][step=1cm]

\usepackage{siunitx}
\usepackage{xmpmulti}
\usepackage[export]{adjustbox}
\usepackage{pifont}% http://ctan.org/pkg/pifont
\newcommand{\cmark}{\ding{51}}%
\newcommand{\xmark}{\ding{55}}%

\usepackage[outline]{contour}
\usepackage{tikz}
\usetikzlibrary{shapes.geometric, arrows}
\usetikzlibrary{positioning}

\definecolor{bvtitlecolor}{rgb}{0.98, 0.92, 0.84}
\definecolor{bvoutline}{rgb}{0.1, 0.1, 0.1}

\renewcommand{\bvtitleauthor}{Brett Viren}
\renewcommand{\bvtit}{Bee 2.0 and Wire Cell Toolkit Future Plans}
\renewcommand{\bvtitle}{\LARGE Bee 2.0 and Wire Cell Toolkit \\ Future Plans}
\renewcommand{\bvevent}{Wire Cell Summit 7-9 Dec 2015}
\renewcommand{\bvbeamerbackground}{}

% http://tex.stackexchange.com/a/23550
\makeatletter
\providecommand{\beamer@slideinframe}{0}%
\tikzset{highlight/.code={\ifnum#1=\beamer@slideinframe \tikzset{draw=red,text=red}\fi},highlight/.value required}
\makeatother

\setbeamertemplate{headline}{%
\leavevmode%
  \hbox{%
    \begin{beamercolorbox}[wd=\paperwidth,ht=2.5ex,dp=1.125ex]{palette quaternary}%
    \insertsectionnavigationhorizontal{\paperwidth}{}{\hskip0pt plus1filll}
    \end{beamercolorbox}%
  }
}
\begin{document}
\input{beamer/title.tex}
\input{beamer/toc.tex}

\lstset{%
  language=C++,
  basicstyle=\scriptsize\ttfamily,
  keywordstyle=\color{blue}\ttfamily,
  stringstyle=\color{red}\ttfamily,
  commentstyle=\color{gray}\ttfamily,
  morecomment=[l][\color{magenta}]{\#}
}

\section{Bee 2.0}

\subsection{Concepts}

\begin{frame}
  \frametitle{Assumptions}
  
  Want a tool to inject ad-hoc, human invented heuristics into the reconstruction.

  \vspace{3mm}
  Assumptions:
  \begin{itemize}
  \item[$\rightarrow$] current automated algorithms are not fully adequate.
  \item[$\rightarrow$] improving them by writing code is too slow.
  \item[$\rightarrow$] want some full reconstruction results quickly.
  \item[$\rightarrow$] can ``port'' human heuristics to new automated code.
  \end{itemize}

\end{frame}

\begin{frame}
  \frametitle{Analogy}
  \begin{itemize}
  \item Doctors get images from X-ray, CT scan, MRI
    scans, etc.
  \item Researchers build an automated computer
    program to tell what diseases the patient has,
    based on the features “seen” in the images.
    \begin{itemize}
    \item False positives might be high
    \item  False negatives might be high
    \item Hospitals can not afford to have high probability of
      mistakes
    \end{itemize}
  \item In the end still need expert doctors’ consultations
    to make final decisions
  \end{itemize}

\end{frame}

\begin{frame}
  \frametitle{What are the goals?}
  What we have now:
  \begin{itemize}
  \item Very robust 3-D images.
  \item View images by human with Bee 1.0.
  \end{itemize}

  The current steps:
  \begin{itemize}
  \item Automated pattern recognition (of some efficacy).
  \item Views of the resulting objects (eg clusters)
  \end{itemize}

  Questions:
  \begin{itemize}
  \item[$\rightarrow$]  What are the metrics by which to judge the performance?
  \item[$\rightarrow$]  What is the goal to reach
  \item[$\rightarrow$]  How can we improve the performance to reach the goal?
  \end{itemize}

\end{frame}

\begin{frame}
  \frametitle{Human vs. Machine}
  Humans are very good at pattern recognition
  \begin{itemize}
  \item[$\rightarrow$] By eye, can easily judge if and where the reconstruction
went wrong.
  \item[$\rightarrow$] Can we inject such kind of knowledge into the
reconstruction?
  \end{itemize}

  \vfill

  Machines are very good at processing a lot of
  data and following our instructions.
  \begin{itemize}
  \item[$\rightarrow$] Can we help the machine to learn?
  \end{itemize}
\end{frame}

\begin{frame} 
  \frametitle{Human-directed automatic reconstruction}
  \begin{enumerate}
  \item Human operates on \textbf{selectable graphical elements}, eg:
    \begin{itemize}\footnotesize
    \item Add or remove cells from a blob or cluster.
    \item Join or break clusters or tracks.
    \item Change pattern recognition category.
    \item Associate clusters to particles (eg, two showers to a $\pi^0$)
    \end{itemize}
  \item Bee sends \textit{command object} to backend.
  \item Database records command and associated initial state.
  \item Backend applies command via Wire Cell and reruns automated reconstruction as needed.
  \item Database records new state, associated with command.
  \item Notify user when done:
    \begin{itemize}\footnotesize
    \item Bee refreshes event if results are fast, or send email/SMS
      with new link.
    \end{itemize}
  \end{enumerate}
  ``Physics PhotoShop''
\end{frame}

\begin{frame}
  \frametitle{Mining Heuristics}
  
  Command database has secondary use:
  \begin{itemize}\footnotesize
  \item Allows for full reproduction/replay of operations later.
  \item Further, can we mine database for things to feed back to automated code?
  \item[$\rightarrow$] Frequent human operations may indicate a heuristic which is ripe for invention.
  \item Maybe this metadata is useful input into some machine learning (suggested to us)
  \end{itemize}

  \vfill

  ``Google Analytics''
\end{frame}

\subsection{Design}
\begin{frame}
  \tableofcontents[currentsection,hideothersubsections]
\end{frame}

\begin{frame}
  \frametitle{Front-end requirements}
  \begin{itemize}
  \item Show necessary diagnostic plots/summaries to
help the human make decisions.
  \item Pick/select possible objects
    \begin{itemize}
    \item Hits, clusters, tracks, vertices, etc.
    \end{itemize}
  \item Form possible operational command objects
    \begin{itemize}
    \item Add, delete, split, merge, etc.
    \end{itemize}
  \item Send operation states to a service.
  \item Get new results back from the service.
  \item Live-update the new results in the web browser.
  \item Support Undo / Revert.
  \end{itemize}
\end{frame}

\begin{frame}
  \includegraphics[width=\textwidth,trim=25mm 16cm 35mm 25mm,clip]{bee2-ecosystem.pdf}
\end{frame}
\begin{frame}
  \frametitle{Bee 2.0 Design Highlights}

  Structure overall system async Pub/Sub \& RPC (eg,  WAMP).
  \begin{itemize}\footnotesize
  \item A networked crossbar with support for JS, Python, etc.
  \item Provides flexible ways to return possibly high latency results.
  \item Assures capture of all input and results.
  \item Focus on overall system logic not gory details of protocols.
  \end{itemize}

  Bee frontend:
  \begin{itemize}\footnotesize
  \item Support more reconstructed object types (clusters, tracks, showers)
  \item Add selection of entities, forming of command objects.
  \item User registration UI.
  \item WAMP hooks (command object/results/auth).
  \end{itemize}

  Bee backend
  \begin{itemize}\footnotesize
  \item May still keep django for serving Bee JS, event indices, handling user uploads.
  \end{itemize}

  Wire Cell batch service
  \begin{itemize}\footnotesize
  \item Respond to RPC for applying commands.
  \item Manage job queue running on dedicated cluster.
  \item Publish results as they come in.
  \end{itemize}

\end{frame}


\section{Toolkit}

\begin{frame}
  \tableofcontents[currentsection,hideothersubsections]
\end{frame}

\subsection{Roadmap}
\begin{frame}
  \frametitle{Wire Cell Toolkit Roadmap}

  \begin{enumerate}\small
  \item Complete design and initial implementation for Data Flow
    Programming (DFP) execution model based on Intel TBB.
  \item Port prototype algorithms into toolkit.
  \item Evaluate performance on single-machine, multi-thread.
  \item Investigate GPU-accelerated DFP nodes for bottlenecks.
  \item Develop a Wire Cell service application.
  \item Implement distributed DFP (ie MPI/ZMQ-edges).
  \item Port toolkit to run on HPC (target ANL's Mira)
  \end{enumerate}

  Once \#1 is done the rest can proceed, more or less, in parallel.

  \vfill

  \textbf{Volunteers?}

  \vfill

  I'll go through what each task needs $\longrightarrow$

\end{frame}

\subsection{Tasks}

\begin{frame}
  \frametitle{Finish DFP}
  \begin{itemize}
  \item There is a clash between OO and GP to contend with.
  \item Dynamic plugins, NamedFactory, Interfaces are all OO.
  \item TBB is GP (so are others like Boost.Pipeline).
  \item To marry these two paradigms requires a lot of scaffolding (or a better brain than I have).
  \item Expert input on this very technical detail would be most welcome.
  \item[$\rightarrow$] if you have experience/input, let's talk offline!
  \end{itemize}
\end{frame}

\begin{frame}
  \frametitle{Port prototype algorithms}

  \begin{itemize}\small
  \item Xin has produced a \textbf{huge body of intellectual work} in the prototype.
  \item Focus was on results, not maintenance or software design.
  \item Effort is needed to \textbf{understand} each algorithm and \textbf{refactor} its code into Toolkit concepts.
  \item \textbf{Porters} will contend with:
    \begin{itemize}\footnotesize
    \item Extracting hard-coded configuration parameters.
    \item Monolithic code blocks need to be split.
    \item Adapting to data model changes.
    \item Conversion of statefull classes to functional ones.
    \item Go from ``services'' to dependency injection.
    \end{itemize}
  \item Xin has done some \textbf{optimization}, but I expect more can be had
    along the way.  Porters should be/get familiar with profiling
    tools.
  \end{itemize}
  Suggest to focus on \textbf{reproducing prototype results} before making novel improvements.
\end{frame}

\begin{frame}
  \frametitle{Performance Evaluation and Improvements}

  \begin{itemize}
  \item Will be done, in part, during the porting exercise.
  \item But, want a \textbf{systematic performance audit} of full chain.
  \item Understand performance as a function of thread count.
  \item \textbf{Kill any easy bottlenecks}.
  \item Document before/after performance and the changes.
  \end{itemize}

  Maybe not a lot of fun, but the results will make you a hero!

\end{frame}

\begin{frame}
  \frametitle{GPU Acceleration}
  \begin{itemize}
  \item Best done after CPU-level optimization is exhausted.
  \item Understand remaining bottlenecks.
  \item Determine if they can be accelerated with GPU.
  \item Develop drop-in replacements nodes which run alg on GPU instead of CPU.
  \end{itemize}

  This work should be done with awareness of existing and emerging GPU
  and Phi use on HPC (eg, the new BNL HPC).

\end{frame}

\begin{frame}
  \frametitle{Wire Cell Service}

  Two driving applications:
  \begin{description}
  \item[Bee 2.0] need prompt processing backend services which driven
    by queries from user on a web browser
  \item[HPC] (or even Grid/batch) backend driven by queries from a
    LArSoft client module with Wire Cell processes running on low
    RAM/core but multi/massive-threaded servers.
  \end{description}

  \begin{itemize}
  \item Want reusable components for both applications.
  \item Tie-ins with next task.
  \item Best to have prior experience in developing network servers.
  \item Need to pick a suitable client/server protocol that can be
    implemented on HPC and can mesh well with C++ and JavaScript
    clients.  Or just use WAMP like with Bee 2.0.
  \end{itemize}

  If this is interesting to you?  Talk with Chao and me.

\end{frame}

\begin{frame}
  \frametitle{Distributed Wire Cell}

  \begin{itemize}
  \item Based on Performance Evaluation task, determine if
    multi-machine parallelism is even required.
  \item Implement by extending the DFP functionality to support
    DFP-node connections across the network.
  \item Need to select one (or better yet allow multiple) network
    communication methods (MPI, ZeroMQ, WAMP or ???).
  \item Should work in conjunction with Wire Cell Service work.
  \end{itemize}

  As noted, this task is speculative.  We may not need it.

\end{frame}

\begin{frame}
  \frametitle{Wire Cell on HPC}
  \small
  \begin{itemize}
  \item Based on Performance Evaluation task, determine if HPC is
    required or if Wire Cell can keep up on Grid resources alone.
  \item Porting to HPC requires understanding limitations specific
    to each target HPC:
    \begin{description}\footnotesize
    \item [Architectural] eg, ROOT doesn't work, dynamic linking
      considered harmful.
    \item [Environmental] eg, special ``edge'' computing to run LArSoft and/or data source/sink.
    \item [Security] HPCs less accessible than Grid
      nodes.  How to get \textbf{high data rate} (full raw data) in and results out?  
    \item [Other] HPC is largely new to neutrino physicists, expect interesting challenges.
    \end{description}
  \end{itemize}

  \vspace{-3mm}
  Initial efforts:
  \begin{itemize}\footnotesize
  \item Already collaborating with BNL ``HPC Code Center''.
  \item BNL has a small, 8 node HPC with GPU for testing now and is
    building a larger, modern HPC in the coming year.
  \item Started very useful discussions with ANL to use Mira.
    \begin{itemize}\scriptsize
    \item
      \href{http://www.alcf.anl.gov/events/getting-started-alcf-resources}{link
        to ANL HPC Training courses}
    \end{itemize}
  \end{itemize}

\end{frame}

\section{Summary}

\begin{frame}
  \tableofcontents[currentsection,hideothersubsections]
\end{frame}

\begin{frame}
  \frametitle{Summary}
  \begin{itemize}
  \item Bee 2.0 is an ambitious system allowing \textbf{human-directed automatic reconstruction} with follow-on data mining of applied commands.
  \item Some technical coding/design issues to do but, the \textbf{toolkit is soon ready} for outside contributions.
  \item A number of known tasks are identified and which can largely be worked on in parallel.
  \item \textbf{Volunteers are most welcome!}
    \begin{itemize}
    \item[$\rightarrow$] if interested, let's discuss!
    \end{itemize}
  \item In general, Wire Cell Toolkit can soon serve as a vehicle to
    advance \textbf{new reconstruction ideas} and apply toward \textbf{detector optimization studies}.
  \end{itemize}
\end{frame}

\end{document}

%%% Local Variables:
%%% mode: latex
%%% TeX-master: t
%%% End:
