

% TEXINPUTS=.:$HOME/git/bvtex: latexmk  -pdf <main>.tex
\documentclass[xcolor=dvipsnames]{beamer}

\input{defaults}
\input{beamer/preamble}

\setbeamertemplate{navigation symbols}{}
% \setbeamertemplate{background}[grid][step=1cm]

\usepackage{siunitx}
\usepackage{xmpmulti}
\usepackage[export]{adjustbox}
\usepackage{pifont}% http://ctan.org/pkg/pifont
\newcommand{\cmark}{\ding{51}}%
\newcommand{\xmark}{\ding{55}}%

\usepackage[outline]{contour}
\usepackage{tikz}
\usetikzlibrary{shapes.geometric, arrows}
\usetikzlibrary{positioning}

\definecolor{bvtitlecolor}{rgb}{0.98, 0.92, 0.84}
\definecolor{bvoutline}{rgb}{0.1, 0.1, 0.1}

\renewcommand{\bvtitleauthor}{Brett Viren}
\renewcommand{\bvtit}{Wire Cell Toolkit \\ Future Plans}
\renewcommand{\bvtitle}{\LARGE Wire Cell Toolkit / Future Plans}
\renewcommand{\bvevent}{Wire Cell Summit 7-9 Dec 2015}
\renewcommand{\bvbeamerbackground}{}

% http://tex.stackexchange.com/a/23550
\makeatletter
\providecommand{\beamer@slideinframe}{0}%
\tikzset{highlight/.code={\ifnum#1=\beamer@slideinframe \tikzset{draw=red,text=red}\fi},highlight/.value required}
\makeatother

\setbeamertemplate{headline}{%
\leavevmode%
  \hbox{%
    \begin{beamercolorbox}[wd=\paperwidth,ht=2.5ex,dp=1.125ex]{palette quaternary}%
    \insertsectionnavigationhorizontal{\paperwidth}{}{\hskip0pt plus1filll}
    \end{beamercolorbox}%
  }
}
\begin{document}
\input{beamer/title.tex}
\input{beamer/toc.tex}

\lstset{%
  language=C++,
                basicstyle=\scriptsize\ttfamily,
                keywordstyle=\color{blue}\ttfamily,
                stringstyle=\color{red}\ttfamily,
                commentstyle=\color{gray}\ttfamily,
                morecomment=[l][\color{magenta}]{\#}
}

\section{Roadmap}

\begin{frame}
  \frametitle{General Wire Cell Toolkit Roadmap}

  \begin{enumerate}\small
  \item Complete design and initial implementation for Data Flow
    Programming (DFP) execution model based on Intel TBB.
  \item Port prototype algorithms into toolkit.
  \item Evaluate performance on single-machine, multi-thread.
  \item Investigate GPU-accelerated DFP nodes for bottlenecks.
  \item Develop a Wire Cell service application.
  \item Implement distributed DFP (edges with MPI or ZeroMQ or ???).
  \item Port toolkit to run on HPC (ANL Mira)
  \end{enumerate}

  Once \#1 is done the rest can proceed in parallel.

  \textbf{Any volunteers?}

  \begin{center}
      In general:   \textbf{hack new ideas and do studies!}
  \end{center}


\end{frame}

\section{Tasks}

\begin{frame}
  \tableofcontents[currentsection,hideothersubsections]
  I'll go through what each task needs $\longrightarrow$
\end{frame}

\begin{frame}
  \frametitle{Finish DFP}
  \begin{itemize}
  \item There is a clash between OO and GP to contend with.
  \item Dynamic plugins, NamedFactory, Interfaces are all OO.
  \item TBB is GP (so are others like Boost.Pipeline).
  \item To marry these two paradigms requires a lot of scaffolding (or a better brain than I have).
  \item Expert input on this very technical detail would be most welcome.
  \item[$\rightarrow$] if you have experience/input, let's talk offline!
  \end{itemize}
\end{frame}

\begin{frame}
  \frametitle{Port prototype algorithms}

  \begin{itemize}\small
  \item Xin has produced a huge body of intellectual work in the prototype.
  \item Focus was on results, not maintenance or software design.
  \item Structure supporting the prototype is now different in the toolkit.
  \item Effort is needed to understand each algorithm and refactor its code into Toolkit concepts.
  \item Porters will contend with:
    \begin{itemize}
    \item Extracting hard-coded configuration parameters.
    \item Monolithic code blocks need to be split.
    \item Adapting to data model changes.
    \item Conversion of statefull classes to functional ones.
    \item Go from ``services'' to dependency injection.
    \end{itemize}
  \item Xin has done some optimization, but I expect more can be had
    along the way.  Porters should be/get familiar with profiling
    tools.
  \item Suggest to focus on reproduce prototype results before making your own improvements.
  \end{itemize}
\end{frame}

\begin{frame}
  \frametitle{Performance Evaluation and Improvements}

  \begin{itemize}
  \item Will be done, in part, during the porting exercise.
  \item But, want a \textbf{systematic performance audit} of full chain.
  \item Understand performance as a function of thread count.
  \item \textbf{Kill any easy bottlenecks}.
  \item Document before/after performance and what the fixes were.
  \end{itemize}

  Maybe not a lot of fun, but the results will make you a hero.

\end{frame}

\begin{frame}
  \frametitle{GPU Acceleration}
  \begin{itemize}
  \item Best done after CPU-level optimization is exhausted.
  \item Understand remaining bottlenecks.
  \item Determine if they can be accelerated with GPU.
  \item Develop drop-in replacements for DFP nodes which are
    bottlenecks and which run the algorithm on a GPU.
  \end{itemize}

  This work should be done with awareness of existing and emerging GPU
  use on HPC (eg, the new BNL HPC).

\end{frame}

\begin{frame}
  \frametitle{Wire Cell Service}

  Two driving applications:
  \begin{description}
  \item[Bee 2.0] need prompt processing backend services which driven
    by queries from user on a web browser
  \item[HPC] (or even Grid/batch) backend driven by queries from a
    LArSoft client module with Wire Cell processes running on low
    RAM/core but multi/massive-threaded servers.
  \end{description}

  \begin{itemize}
  \item Want reusable components for both applications.
  \item Tie-ins with next task.
  \item Best to have prior experience in developing network servers.
  \item Need to pick a suitable client/server protocol that can be
    implemented on HPC and can mesh well with C++ and JavaScript
    clients.
  \end{itemize}

  If this is interesting to you, talk with Chao and me.

\end{frame}

\begin{frame}
  \frametitle{Distributed Wire Cell}

  \begin{itemize}
  \item Based on Performance Evaluation task, determine if
    multi-machine parallelism is even required.
  \item Implement by extending the DFP functionality to support
    DFP-node connections across the network.
  \item Need to select one (or better yet allow multiple) network
    communication methods (MPI, ZeroMQ, WAMP or ???).
  \item Should work in conjunction with Wire Cell Service work.
  \end{itemize}

  As noted, this task is speculative.  We may not need it.

\end{frame}

\begin{frame}
  \frametitle{Wire Cell on HPC}
  \small
  \begin{itemize}
  \item Based on Performance Evaluation task, determine if HPC is
    required or if Wire Cell can keep up on Grid resources alone.
  \item Porting to HPC requires understanding limitations specific
    to each target HPC:
    \begin{description}\footnotesize
    \item [Architectural] eg, ROOT doesn't work, dynamic linking
      considered harmful.
    \item [Environmental] eg, special ``edge'' computing to run LArSoft.
    \item [Security] HPCs less accessible than Grid
      nodes.  How to get \textbf{high data rate} in and results out?  
    \item [Other] HPC is largely new to neutrino physicists, expect interesting challenges.
    \end{description}
  \end{itemize}

  Initial efforts:
  \begin{itemize}\footnotesize
  \item Already collaborating with BNL ``HPC Code Center''.
  \item BNL has a small, 8 node HPC with GPU for testing now and is
    building a larger, modern HPC in the coming year.
  \item Started very useful discussions with ANL to use Mira.
  \end{itemize}

\end{frame}

\section{Summary}

\begin{frame}
  \tableofcontents[currentsection,hideothersubsections]
\end{frame}

\begin{frame}
  \frametitle{Summary}
  \begin{itemize}
  \item Once some technical coding/design issues are finished, the toolkit is ready for outside contributions.
  \item A number of known tasks are identified and which can largely be worked on in parallel.
  \item \textbf{Volunteers are most welcome!}
    \begin{itemize}
    \item[$\rightarrow$] if interested, let's discuss!
    \end{itemize}
  \item In general, Wire Cell Toolkit can soon serve as a vehicle to
    \textbf{advance new reconstruction ideas} and apply toward \textbf{detector optimization studies}.
  \end{itemize}
\end{frame}

\end{document}

%%% Local Variables:
%%% mode: latex
%%% TeX-master: t
%%% End:
