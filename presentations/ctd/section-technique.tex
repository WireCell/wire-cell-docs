\section{Wire Cell Technique}

\begin{frame}
  \tableofcontents[currentsection,hideothersubsections]
\end{frame}


\begin{frame}[fragile]
  \frametitle{Wire Cell Reconstruction Method}
  \setbeamercovered{transparent}
  \begin{columns}
    \begin{column}{0.7\textwidth}
      Four main parts:
      \begin{enumerate}
      \item<2> Data Preparation
        \begin{itemize}        \scriptsize
        \item Deconvolve \textbf{detector response and filter noise}.
        \item Construct \textbf{wire} geometry and associated \textbf{cells}.
        \item Form \textbf{time slices} across all wire signal waveforms.
        \end{itemize}
      \item<3> Imaging of Activity
        \begin{itemize}        \scriptsize
        \item The heart of the Wire Cell technique.
        \item Identify regions (``\textbf{cells}'') in each time slice that likely contain the drifted charge.
        \end{itemize}
      \item<4> Pattern Recognition 
        \begin{itemize}         \scriptsize
        \item \textbf{Cluster} imaged activity in space and time slices.
        \item \textbf{Categorize} patterns as track, shower, etc.
        \end{itemize}
      \item<5> Physics Quantities
        \begin{itemize}         \scriptsize
        \item Determine particle ID and kinematics of tracks/showers.
        \end{itemize}
      \end{enumerate}
    \end{column}
    \begin{column}{0.3\textwidth}
      \begin{center}
        \vspace{-10mm}
        \resizebox{!}{\textheight}{
\begin{tikzpicture}[align=center, node distance=2cm]

\tikzstyle{dataobj} = [rectangle, rounded corners, minimum width=3cm, minimum height=1cm,text centered, draw=black, fill=blue!30]
\tikzstyle{process} = [rectangle, minimum width=3cm, minimum height=1cm, text centered, draw=black, fill=orange!30]
\tikzstyle{decision} = [diamond, minimum width=3cm, minimum height=1cm, text centered, draw=black, fill=green!30]
\tikzstyle{arrow} = [thick,->,>=stealth]

\node (daqmc) [process,highlight=2] {DAQ/MC sim};
\node (frames) [process,highlight=2, below of=daqmc] {Signal Processing};
\node (slicing) [process,highlight=2, below of=frames] {Slicing};
\node (slices) [dataobj, below of=slicing] {Time Slices};
\node (wires) [dataobj, left=0.5cm of frames] {Wire geometry};
\node (tiling) [process,highlight=2, left=0.5cm of slicing] {Cell Tiling};
\node (cells) [dataobj, left=0.5cm of slices] {Cells geometry};
\node (hitcells) [process,highlight=3, below of=slices] {Hit cell selection};
\node (blobbing) [process,highlight=3, below of=hitcells] {Blob forming};
\node (clustering) [process,highlight=4, below of=blobbing] {Clustering};
\node (tracking) [process,highlight=4, below of=clustering] {Tracking};
\node (pid) [process,highlight=5, below of=tracking] {Particle ID};
\node (physics) [dataobj, below of=pid] {Physics Quantities};

\draw [arrow] (wires) -- (tiling);
\draw [arrow] (tiling) -- (cells);
\draw [arrow] (cells) -- (hitcells);
\draw [arrow] (daqmc) -- (frames);
\draw [arrow] (frames) -- (slicing);
\draw [arrow] (slicing) -- (slices);
\draw [arrow] (slices) -- (hitcells);
\draw [arrow] (hitcells) -- (blobbing);
\draw [arrow] (blobbing) -- (clustering);
\draw [arrow] (clustering) -- (tracking);
\draw [arrow] (tracking) -- (pid);
\draw [arrow] (pid) -- (physics);
\end{tikzpicture}}
      \end{center}
    \end{column}
  \end{columns}

\end{frame}

\subsection{Data Preparation}


\begin{frame}[fragile]
  \frametitle{Time Slicing}
  
  \begin{columns}
    \begin{column}{0.35\textwidth}
      \begin{center}
        \vspace{-.5cm}

        \includegraphics[width=\textwidth]{slice.pdf}

        \vspace{-2cm}

        \includegraphics[width=\textwidth,trim=0cm 10cm 0cm 0cm,clip]{slice-3D.pdf}

        \scriptsize Focus on one \textbf{time slice} along a plane transverse to the drift.
      \end{center}
    \end{column}
    \begin{column}{0.65\textwidth}

      \includegraphics[width=0.9\textwidth]{wires-and-true-hits.png}

      \begin{itemize} \scriptsize
      \item Slice duration is chosen to \textbf{match
          electronics shaping time}: combine 4 FADC ``ticks'' =
        \SI{2}{\micro\second}.
      \item Select \textcolor{red}{wires} above threshold in the slice.
      \item Identify regions near triple-wire overlap as potential ``\textcolor{blue}{cells}'' holding drifted charge in that \textbf{time slice}.
      \end{itemize}
    \end{column}
  \end{columns}

\end{frame}

\subsection{Imaging Of Activity}

\begin{frame}[fragile]
  \frametitle{Tiling}

  \vspace{-10mm}

  \begin{center}
    \scriptsize Zoom in on the wires and their associated (constructed) cells.

    \includegraphics[width=0.7\textwidth,trim=8.6cm 10cm 8.6cm 9cm,clip]{test_boundcells_uboone.pdf}

    MicroBooNE geometry, grey wires, colored cells.
  \end{center}

  \footnotesize
  \begin{itemize}
  \item Try to localize drifted charge into 2D bins or ``cells'' in
    the plane transverse to the electron drift direction.
  \item Each cell covers region near the approximate
    \textbf{triple-crossing of one wire from each plane}.
  \item Pattern determined by \textbf{pitch}, \textbf{angle} and \textbf{phase} of wire planes.
    \begin{description}
    \item[MicroBooNE] regular isosceles triangles of fixed size.
    \item[DUNE] variety of polygons with a spectrum of sizes.
    \end{description}
  \end{itemize}

  \textbf{The heart of the Wire Cell concept:} if all three
  \textbf{wires} are above threshold in a \textbf{time slice}, the
  associated \textbf{cell} \textit{likely} contains some 
  \textbf{drifted charge}.

\end{frame}

\begin{frame}[fragile]
  \frametitle{Ideal Solution}

  In a \textbf{perfect detector}, simply invert matrix equation:
  \[\vec{w} = \mathbf{G} \vec{c}\]

  \begin{description}\footnotesize
  \item[$\vec{w}$] measured charge on \textbf{wires} (in time slice)
  \item[$\vec{c}$] expected charge o n \textbf{cells} (in time slice)
  \item[$\mathbf{G}$] fixed wire-cell \textbf{connection} matrix (geometry)
  \end{description}

  \onslide<2->{
  In a \textbf{real detector}:

  \begin{itemize}
  \item \textbf{Solution is often ambiguous} $(N_{wires} < N_{cells})$
    \begin{itemize}\footnotesize
    \item CT scanners have 100s of views in each slice, not just 3!
    \end{itemize}
  \item \textbf{Uncertainty in charge measurements} ($\vec{w}$)
    \begin{itemize}\footnotesize
    \item Environmental, electronic and thermal noise.
    \item Statistical uncertainty due to digitization.
    \item Systematic uncertainties from deconvolution.
    \end{itemize}
  \end{itemize}}
\end{frame}


\begin{frame}
  \frametitle{Ambiguities - Example Hit Pattern}

  \begin{columns}
    \begin{column}{0.35\textwidth}
      \begin{center}
        \scriptsize Zoom in on $5 \times 5 \times 3$ wires:
        \includegraphics[width=\textwidth,trim=3.5cm 6cm 5cm 3cm,clip]{example-hit-cells.pdf}        
      \end{center}
    \end{column}
    \begin{column}{0.65\textwidth}
      Cartoon of activity in one time slice:
      \begin{itemize}\footnotesize
      \item 9 cells and their associated wires.
      \item wires: color=plane, thickness=charge
      \item purple shows true drifted charge
      \end{itemize}

      \onslide<2->{
      \begin{description}\footnotesize
      \item[Good] wire \textcolor{blue}{v3} measures no charge, \\$\therefore$ all its cells must \textbf{not} be hit.
      \item[Bad] wires hit by charge in \textbf{c2}, \textbf{c7} and \textbf{c8} induce ``ghost'' at \textbf{c4}.
      \item[Ambiguous] multiple cells measured by same wire.
        How much charge is in \textbf{c6}???
      \end{description}}

      \vfill

      \onslide<3->{\footnotesize\flushright $\Rightarrow$Fight ambiguity by reducing number of unknowns!}
    \end{column}
  \end{columns}
\end{frame}


\begin{frame}
  \frametitle{Blobs}

  \begin{columns}
    \begin{column}{0.6\textwidth}
      \textbf{Reduce matrix size} and (try to) \\
      \textbf{remove ambiguity} by exploiting knowledge
      of cell-nearest-neighbors.
    \end{column}
    \begin{column}{0.4\textwidth}
      \includegraphics[width=0.8\textwidth]{wires-and-true-hits.png}          
    \end{column}
  \end{columns}

  \begin{center}
    cells $\to$ blobs

    wires $\to$ merged wires
  \end{center}

  Same basic problem to solve:
  \[\vec{w_b} = \mathbf{G_{wb}} \vec{b}\]

  \begin{itemize}
  \item \textbf{Sometimes} gives more favorable numerology: $N_{blob} \lesssim N_{w_b}$
  \item \textbf{But not always!} some $\mathbf{G_{wb}}$ can still not be inverted.
  \item[$\rightarrow$] And, we still haven't taken into account \textbf{uncertainty}.
  \end{itemize}
\end{frame}

\begin{frame}
  \frametitle{Charge Measurement Uncertainty}

  Can form:

  \[\chi^2 = (\vec{w}_{meas}-\vec{w}_{exp})^\intercal\mathbf{V}^{-1} (\vec{w}_{meas}-\vec{w}_{exp})\]

  \vspace{3mm}

  \begin{description}\footnotesize
  \item[$\vec{w}_{meas}$] measured (merged) wire charges (in a slice)
  \item[$\vec{w}_{exp}$] expected wire charge $(\vec{w}_{exp} = \vec{w}_b = \mathbf{G_{wb}}\vec{b})$
  \item[$\mathbf{V}$] wire charge uncertainty covariance matrix
  \end{description}

  \vspace{3mm}

  \footnotesize{Minimization over blob charges is equivalent to matrix inversion and
  gives:}

  \[\vec{b} = (\mathbf{G}^\intercal\mathbf{V}\mathbf{G})^{-1}\mathbf{G}^\intercal\mathbf{V}^{-1}\mathbf{B}\vec{w}_{meas} \]

  \footnotesize
  \begin{description}
  \item[$\mathbf{B}$] matrix connects ``merged'' wires to individual wires (per slice).
  \end{description}

  \begin{center}
    Reminder: only works if $N_{blob} < N_{w_b}$!
  \end{center}
\end{frame}

\begin{frame}
  \frametitle{Remaining Ambiguities}

  Even after forming \textbf{blobs} sometimes still some \textbf{time slices} with:
  \[N_{blob} > N_{w_b}\]

  $\Rightarrow$ \textbf{zero-value eigenvalues}, remaining ambiguity.


  \begin{itemize}
  \item<2-> Try removing \textbf{blobs in
      combinations}: 2, 3, ... until all \textbf{eigenvalues are nonzero}

  \item<2-> \textbf{Combinatorics $\Rightarrow$ hugely time consuming!}  
    \begin{itemize}
    \item Algorithm optimization and shortcuts employed.
    \item Still, some time slices can take $\sim$hours.
    \item[$\Rightarrow$] We are looking into exploiting GPUs.
    \end{itemize}
  \item<2->  For now, if no solution is found after fixed number of iterations $\Rightarrow$ give up on time slice....
  \end{itemize}
\end{frame}


\begin{frame}[fragile]
  \frametitle{The Payoff: imaged \SI{3}{\giga\electronvolt} $\nu_e$ interaction}
  
  \begin{columns}
    \begin{column}{0.5\textwidth}
      \begin{center}
        \includegraphics[width=\textwidth,trim=3cm 10cm 3cm 10cm,clip]{payoff-true.png}

        True energy depositions.
      \end{center}
    \end{column}
    \begin{column}{0.5\textwidth}
      \begin{center}
        \includegraphics[width=\textwidth,trim=3cm 10cm 3cm 10cm,clip]{payoff-reco.png}

        Wire Cell Imaging.
      \end{center}
    \end{column}
  \end{columns}

  \vfill

  \footnotesize
  \begin{itemize}
  \item<2-> \textbf{Excellent imaging} of major features and isolated
    activity.
    \begin{itemize}\scriptsize
    \item[$\rightarrow$] a static 2D view doesn't do it justice!  \href{http://www.phy.bnl.gov/wire-cell/bee/set/6/event/0/}{Follow link to view it online}.
    \end{itemize}
  \item<3-> \textbf{Residual ambiguity} seen as wide blue patches.
    \begin{itemize}\scriptsize
    \item[$\rightarrow$] \textbf{Inherent problem of tomography using low number of viewing angles}
    \item[$\rightarrow$] Will pursue an \textbf{iterative} approach:
      constrain ambiguous regions after reconstructing the good parts
      to the kinematics-level.
    \end{itemize}
  \end{itemize}
\end{frame}


\subsection{Pattern Recognition}

\begin{frame}
  \frametitle{Post-imaging Pattern Recognition}

  \begin{enumerate}
  \item \textbf{cluster} together blobs contiguous in space and time
    (slice).
  \item \textbf{track} straight line through a cluster.
  \item \textbf{categorize} track fit solutions/failures.
  \end{enumerate}
  Some initial, coarse categories:
  \begin{description}
  \item[track] cluster is well characterized by the track.
  \item[shower] cluster consistent with an EM/hadronic shower.
  \item[short] cluster appears to be a ``short track'' (eg, $\delta$-ray).
  \item[undefined] no well-suited categorization.
  \end{description}

  \begin{itemize}
  \item This is an \textbf{active area of development} for Wire Cell.
  \item Plan for \textbf{semi-automated, hand-scanning} to \textbf{map out problem space}.
  \item Maybe a problem suited for \textbf{machine learning}? 
  \end{itemize}

\end{frame}
