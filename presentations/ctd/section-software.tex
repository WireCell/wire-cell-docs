\section{Wire Cell Software}

\begin{frame}
  \tableofcontents[currentsection,hideothersubsections]
\end{frame}


\begin{frame}
  \frametitle{Wire Cell Software Ecosystem}

  The software is composed into three main parts:

  \vspace{3mm}

  \begin{description}
  \item[visualization] the ``Bee'' web application (Chao Zhang)
  \item[prototype] reconstruction algorithms, initial proof of
    principle (Xin Qian)
  \item[toolkit] production process, parallelism and basis for long-term development (bv)
  \end{description}

\end{frame}

\subsection{Bee Display}

\begin{frame}
  \frametitle{Bee: an interactive 3D visualization system}

  Select features:
  \begin{itemize}
  \item \textbf{Web browser-based} 3D event display, 
  \item Shows variety of reconstructed and ``true'' information.
  \item Implemented in \textbf{JavaScript/WebGL} (\textbf{Django} backend).
  \item Simple \textbf{JSON} data file format, 
    \begin{itemize}
    \item \href{http://bnlif.github.io/wire-cell-docs/viz/uploads/}{drag-and-drop user
      file uploads}.
    \end{itemize}
  \end{itemize}

  \onslide<2->{ 
    \textbf{Bee 2.0} in development
    \begin{itemize}
    \item[$\rightarrow$] \textbf{semi-automated, human-guided} pattern
      recognition.
    \item[$\rightarrow$] collect and replay human decisions.
    \item[$\rightarrow$] look for \textbf{low-hanging fruit to
        automate}.
    \item[$\rightarrow$] maybe feed into machine learning?
    \end{itemize}}

\end{frame}

\begin{frame}
  \frametitle{Bee Screenshot}
  \begin{center}
    \includegraphics[height=0.7\textheight]{bee-full-gui.png}    
  \end{center}
  \begin{center}
    Try it yourself: \url{http://www.phy.bnl.gov/wire-cell/bee/}
  \end{center}
\end{frame}


\subsection{Prototype}

\begin{frame}
  \frametitle{Wire Cell Working Prototype}
  \footnotesize

  \begin{itemize}
  \item \textbf{Very successful proof of principle!}
  \item Currently \textbf{leads the state of the art} in LArTPC
    reconstruction techniques.
  \end{itemize}

  \begin{center}
    \includegraphics[height=0.5\textheight,trim=0cm 0cm 0cm 2cm,clip]{xin-shower.png}

    \scriptsize
    Example result showing imaging + pattern recognition.  Colors indicate identified tracks and showers.
  \end{center}

\end{frame}

\subsection{Toolkit}

\begin{frame}
  \frametitle{Wire Cell Toolkit}

  \begin{columns}
    \begin{column}{0.65\paperwidth}
    \end{column}
    \begin{column}{0.35\paperwidth}
      \vspace{-20mm}
      \includegraphics[width=\textwidth]{deps.pdf}      
    \end{column}
  \end{columns}

  \vspace{-15mm}

  Address compromises made in the name\\
  of rapid prototyping 
  \begin{itemize}
  \item Modular \textbf{packaging and build} system (waf).
  \item Comprehensive \textbf{API} via abstract interface classes.
  \item Mindful of \textbf{dependencies}, external and internal.
  \item Built-in wire and cell \textbf{geometry descriptions} or load from file.
  \item Includes simple \textbf{LArTPC detector simulation}.
  \item Rewriten implementations of \textbf{prototype algorithms}.
  \item \textbf{Data Flow Programming} execution model with \textbf{abstracted DFP engine}.
  \end{itemize}

  \vfill

  Just now becoming available but with some work still to do.

\end{frame}

\begin{frame}[fragile]
  \frametitle{Wire Cell Execution Model}

  \begin{columns}
    \begin{column}{0.7\textwidth}
      \footnotesize 
      The toolkit supports \textbf{data flow programming} paradigm
      \begin{itemize}
        \item Design influenced by
          \href{http://www0.bnl.gov/events/details.php?q=8932}{VisTrails} and others.
        \item Data flows through a \textbf{graph} made from:
          \begin{description}
          \item[vertices:] \textbf{computational units} (algorithms)
          \item[edges:] \textbf{data queues} of a given type
          \end{description}
        \item Streamed processing minimizes RAM usage.
        \item High-level ``graph programming'' in user config.
        \item \textbf{Thread-safe queues $\Rightarrow$ parallel processing.}
        \item Abstract graph execution machinery.
          \begin{itemize}\scriptsize
          \item Intel TBB provides reference implementation.
          \end{itemize}
        \end{itemize}
      \end{column}
      \begin{column}{0.3\textwidth}

        \vspace{-10mm}

        \includegraphics[width=\textwidth]{dataflow.pdf}

        \tiny One possible Wire Cell flow.
      \end{column}
    \end{columns}
\end{frame}

